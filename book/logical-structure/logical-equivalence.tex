% !TeX root = ../../infdesc.tex
\section{Equivalencia lógica}
\secbegin{secLogicalEquivalence}

Motivamos o conteúdo desta seção com um exemplo.

\begin{example}
\label{exNoEvenPrimeGreaterThanTwo}
Considere as duas fórmulas lógicas a seguir, onde $P$ denota o conjunto de todos os números primos.
\begin{enumerate}[(1)]
\item $\forall n \in P,\, (n > 2 \Rightarrow [\exists k \in \mathbb{Z},\, n=2k+1])$;
\item $\neg \exists n \in P,\, (n > 2 \wedge [\exists k \in \mathbb{Z},\, n=2k])$.
\end{enumerate}
A fórmula lógica (1) se traduz em “todo número primo maior que dois é ímpar”, e a fórmula lógica (2) se traduz em “não existe um número primo par maior que dois”. Estas afirmações são evidentemente \textit{equivalentes}---elas significam a mesma coisa---mas sugerem diferentes estratégias de prova:

\begin{enumerate}[(1)]
\item Corrija um número primo $n$, assuma que $n>2$, e então prove que $n = 2k+1$ para alguns $k \in \mathbb{Z}$.
\item Suponha que exista um número primo $n$ de tal modo que $n>2$ e $n=2k$ para algum $k \in \mathbb{Z}$, e deduza a contradição.
\end{enumerate}

Embora a afirmação (1) traduza mais diretamente a afirmação simples “todo número primo maior que dois é ímpar”, é a estratégia de prova sugerida por (2) que é mais fácil de usar. De fato, se $n$ é a número primo de tal modo que $n>2$ e $n=2k$ para $k \in \mathbb{Z}$, então $2$ é um divisor de $n$ diferente de $1$ e $n$ (desde que $1<2<n$), contradizendo afirmação que $n$ é primo.
\end{example}

A noção de \textit{logical equivalence}, captura precisamente o sentido em que a fórmula lógica em (1) e (2) em \Cref{exNoEvenPrimeGreaterThanTwo} `significa a mesma coisa'. Ser capaz de transformar uma fórmula lógica numa forma diferente (mas equivalente) permite-nos identificar uma gama mais ampla de estratégias de prova viáveis.

\begin{definition}
\label{defLogicalEquivalence}
\index{logical equivalence}
\index{equivalence!logical}
\nindex{logical equivalence}{$\equiv$}{logical equivalence}
Seja $p$ e $q$ fórmulas lógicas. Nós dizemos que $p$ e $q$ são \textbf{logicalmente equivalente}, e escrevemos $p \equiv q$ \inlatex{equiv}\lindexmmc{equiv}{$\equiv$}, se $q$ pode ser derivado de $p$ e $p$ pode ser derivado de $q$.
\end{definition}

\subsection*{Equivalencia lógica de fórmulas proposicionais}

Enquanto \Cref{defLogicalEquivalence} define equivalência lógica entre fórmulas lógicas arbitrárias, começaremos focando nossa atenção na equivalência lógica entre fórmulas \textit{proposicionais}, como vimos em \Cref{secPropositionalLogic}.

Primeiro, vejamos alguns exemplos de como podem ser as provas de equivalência lógica. Esteja avisado: eles não são muito agradáveis ​​de ler! Mas há luz no fim do túnel. Depois de lutar \Cref{exConjunctionDistributesOverDisjunction,exImplicationInTermsOfDisjunction} e \Cref{exPAndQImpliesRIffPImpliesRAndQImpliesR},apresentaremos uma ferramenta muito rápida e fácil para provar que fórmulas proposicionais são logicamente equivalentes.
\begin{example}
\label{exConjunctionDistributesOverDisjunction}
Nós demostramos que p∧(q∨r)≡(p∧q)∨(p∧r)p \wedge (q \vee r) \equiv (p \wedge q) \vee (p \wedge r), onde pp, qq and rr são variáveis proposicionais.

\begin{itemize}
\item Primeiro assuma que p∧(q∨r)p \wedge (q \vee r) é verdade. Então pp é verdade e q∨rq \vee r é verdade pela definição de conjunção. Pela definição de disjunção, qualquer qq é verdade ou rr é verdade.
\begin{itemize}
\item Se qq é verdade, entãi p∧qp \wedge q é verdade pela definição de conjunção.
\item Se $r$ é verdade, então $p \wedge r$ é verdade pela definição de conjunção.
\end{itemize}
Em ambos os casos nós temos que $(p \wedge q) \vee (p \wedge r)$ é verdade pela definição de conjunção.

\item Agora assuma que $(p \wedge q) \vee (p \wedge r)$ é verdade. Então qualquer $p \wedge q$ is true or $p \wedge r$ é verdade, pela definição de disjunção.
\begin{itemize}
\item Se $p \wedge q$ é verdade, então $p$ é veradde e $q$ é verdadeiro pela definição de conjunção.
\item If $p \wedge r$ é verdade, então $p$ é verrdade e $r$ é verdadeiro pela definição de conjunção.
\end{itemize}
Em ambos os casos nós temos que $p$ é verdade, e que $q \vee r$ é verdairo pela definição de disjunção. Por isso $p \wedge (q \vee r)$ é verdadeiro pela definição de conjunção.
\end{itemize}

Já que podemos derivar $(p \wedge q) \vee (p \wedge r)$ from $p \wedge (q \vee r)$ e vice-versa, segue que
\[ p \wedge (q \vee r) \equiv (p \wedge q) \vee (p \wedge r)\]
como requerido.
\end{example}

\begin{example}
\label{exImplicationInTermsOfDisjunction}
Nós provamos que $p \Rightarrow q \equiv (\neg p) \vee q$, onde $p$, $q$ e $r$ são variáveis ​​proposicionais.

\begin{itemize}
\item Primeiro assuma que $p \Rightarrow q$ é verdade. Pela lei do meio excluído (\Cref{axLEM}), qualquer $p$ é verdadeiro ou $\neg p$ é verdadeiro---nós derivamos $(\neg p) \vee q$ em cada caso.
\begin{itemize}
\item Se $p$ é verdade, então ja qué $p \Rightarrow q$ é verdade,segue de \elimrule{\Rightarrow} que $q$ é verdadeiro, e assim $(\neg p) \vee q$ é verdade por \introrulesub{\vee}{2};
\item Se $\neg p$ é verdade, então $(\neg p) \vee q$ é verdadeiro por \introrulesub{\vee}{1}.
\end{itemize}
Em ambos os casos, nós vemos que$(\neg p) \vee q$ é verdade.

\item Agora suponha que $(\neg p) \vee q$ seja verdadeiro. Para provar que $p \Rightarrow q$ é verdadeiro, basta por \introrule{\Rightarrow} assumir que $p$ é verdadeiro e derivar $q$. Portanto, suponha que $p$ seja verdadeiro. Como $(\neg p) \vee q$ é verdadeiro, temos que $\neg p$ é verdadeiro ou $q$ é verdadeiro.
\begin{itemizar}
\item Se $\neg p$ for verdadeiro, então obtemos uma contradição a partir da suposição de que $p$ é verdadeiro, e então $q$ é verdadeiro pelo princípio da explosão (\Cref{axPrincipleOfExplosion}).
\item Se $q$ for verdadeiro\dots{} bem, então $q$ é verdadeiro --- não há mais nada a provar!
\end{itemize}
Em ambos os casos temos que $q$ é verdadeiro. Portanto, $p \Rightarrow q$ é verdadeiro.
\end{itemize}


Nós derivamos $(\neg p) \vee q$ de $p \Rightarrow q$ e vice-versa, e então as duas fórmulas são logicamente equivalentes.
\end{example}

\begin{exercise}
\label{exPAndQImpliesRIffPImpliesRAndQImpliesR}
Seja $p$, $q$ and $r$ ser variáveis ​​proposicionais. Prove que a fórmula proposicional $(p \vee q) \Rightarrow r$ é logicamente equivalente para $(p \Rightarrow r) \wedge (q \Rightarrow r)$.
\end{exercise}

Trabalhar com as derivações cada vez que quisermos provar a equivalência lógica pode se tornar complicado mesmo para pequenos exemplos como\Cref{exConjunctionDistributesOverDisjunction,exImplicationInTermsOfDisjunction} and \Cref{exPAndQImpliesRIffPImpliesRAndQImpliesR}.


O teorema a seguir reduz o problema de provar a equivalência lógica entre fórmulas \textit{proposicionais} à tarefa puramente algorítmica de verificar quando as fórmulas são verdadeiras e quando são falsas em uma lista (relativamente) pequena de casos. Iremos agilizar ainda mais esse processo usando \textit{tabelas verdade} (\Cref{defTruthTable})


\begin{theorem}
\label{thmLogicalEquivalentIffSameTruthValues}
Duas fórmulas proposicionais são logicamente equivalentes se e somente se seus valores verdade são os mesmos sob qualquer atribuição de valores verdade às suas variáveis ​​proposicionais constituintes.
\end{theorem}


\begin{cidea}
Uma prova formal deste facto está um pouco fora do nosso alcance neste momento, embora possamos prová-lo formalmente através  \textit{structural induction}, introduzido em \Cref{secStructuralInduction}.

A ideia da prova é que, uma vez que as fórmulas proposicionais são construídas a partir de fórmulas proposicionais mais simples usando operadores lógicos, o valor verdade de uma fórmula proposicional mais complexa é determinado pelos valores verdade das suas subfórmulas mais simples. Se continuarmos a “perseguir” estas subfórmulas, acabaremos apenas com variáveis ​​proposicionais.

Por examplo, o verdadeiro valor de $(p \Rightarrow r) \wedge (q \Rightarrow r)$ é determinado pelos valores verdadeiros de $p \Rightarrow r$ and $q \Rightarrow r$ de acordo com as regras para o operador de conjunção $\wedge$. Por sua vez, o valor de verdade de $p \Rightarrow r$ é determinado pelos valores de verdade de $p$ and $r$ de acordo com o operador de implicação $\Rightarrow$, e o valor verdadeiro de $q \Rightarrow r$ é determinado pelos valores de verdade de $q$ e $r$ de acordo com o operador de implicação novamente. Segue-se que o valor de verdade de toda a fórmula proposicional $(p \Rightarrow r) \wedge (q \Rightarrow r)$ é determinado pelos verdadeiros valores de $p,q,r$ de acordo com as regras de $\wedge$ and $\Rightarrow$.

Se alguma atribuição de valores de verdade a variáveis ​​proposicionais torna uma fórmula proposicional verdadeira, mas outra falsa, então deve ser impossível derivar uma da outra – caso contrário, obteríamos uma contradição. Portanto, ambas as fórmulas proposicionais devem ter os mesmos valores de verdade, não importa qual atribuição de valores de verdade seja dada às suas variáveis ​​proposicionais constituintes.
\end{cidea}

Desenvolvemos agora uma forma sistemática de verificar os valores verdade de uma fórmula proposicional sob cada atribuição de valores verdade às suas variáveis ​​proposicionais constituintes.

\begin{definition}
\label{defTruthTable}
\index{truth table}
O \textbf{verdairo valor} de uma fórmula proposicional é a tabela com uma linha para cada atribuição possível de valores de verdade às suas variáveis ​​proposicionais constituintes e uma coluna para cada subfórmula (incluindo as variáveis ​​proposicionais e a própria fórmula proposicional). As entradas da tabela verdade são os valores verdade das subfórmulas.
\end{definition}

\begin{example}
\label{exNegationConjunctionDisjunctionImplicationTruthTable}
A seguir estão as tabelas verdade para $\neg p$, $p \wedge q$, $p \vee q$ and $p \Rightarrow q$.

\begin{center}
\begin{tabular}{c|c}
$p$ & $\neg p$ \\ \hline
\TT & \FF \\
\FF & \TT \\
\multicolumn{2}{c}{\phantom{\TT}\phantom{\FF}} \\
\multicolumn{2}{c}{\phantom{\TT}\phantom{\FF}}
\end{tabular}
%
\hspace{15pt}
%
\begin{tabular}{cc|c}
$p$ & $q$ & $p \wedge q$ \\ \hline
\TT & \TT & \TT \\
\TT & \FF & \FF \\
\FF & \TT & \FF \\
\FF & \FF & \FF
\end{tabular}
%
\hspace{15pt}
%
\begin{tabular}{cc|c}
$p$ & $q$ & $p \vee q$ \\ \hline
\TT & \TT & \TT \\
\TT & \FF & \TT \\
\FF & \TT & \TT \\
\FF & \FF & \FF
\end{tabular}
%
\hspace{15pt}
%
\begin{tabular}{cc|c}
$p$ & $q$ & $p \Rightarrow q$ \\ \hline
\TT & \TT & \TT \\
\TT & \FF & \FF \\
\FF & \TT & \TT \\
\FF & \FF & \TT
\end{tabular}
\end{center}
\end{example}

Em \Cref{exNegationConjunctionDisjunctionImplicationTruthTable} nós temos usedo o símbolo \TT{} \inlatex{checkmark}\lindexmmc{checkmark}{$\checkmark$} para significar `verdade' e \FF{} \inlatex{times}\lindexmmc{times}{$\times$} para significar `falso'. Alguns autores adotam outras convenções, tanto quanto $T,F$ or $\top,\bot$ \inlatex{top,\textbackslash{}bot}\lindexmmc{top}{$\top$}\lindexmmc{bot}{$\bot$} or $1,0$ or $0,1$---as posibilidades são infinitas!

\begin{exercise}
Use as definiçõe de $\wedge$, $\vee$ and $\Rightarrow$ para justificar as tabelas verdade em \Cref{exNegationConjunctionDisjunctionImplicationTruthTable}.
\hintlabel{exJustifyBasicTruthTables}{Observe que você pode precisar usar a lei do meio excluído (\Cref{axLEM}) e o princípio da explosão (\Cref{axPrincipleOfExplosion}).}
\end{exercise}

O próximo exemplo mostra como as tabelas verdade para os operadores lógicos individuais (as in \Cref{exNegationConjunctionDisjunctionImplicationTruthTable}) podem ser combinadas para formar uma tabela verdade para uma fórmula proposicional mais complicada que envolve três variáveis ​​proposicionais.
\begin{example}
\label{exFirstExampleOfTruthTable}
A seguir está a tabela verdade para $(p \wedge q) \vee (p \wedge r)$.
\begin{center}
\begin{tabular}{ccc|cc|c}
$p$ & $q$ & $r$ & $p \wedge q$  & $p \wedge r$ & $(p \wedge q) \vee (p \wedge r)$\\ \hline
\TT & \TT & \TT & \TT           & \TT          & \TT \\
\TT & \TT & \FF & \TT           & \FF          & \TT \\
\TT & \FF & \TT & \FF           & \TT          & \TT \\
\TT & \FF & \FF & \FF           & \FF          & \FF \\
\FF & \TT & \TT & \FF           & \FF          & \FF \\
\FF & \TT & \FF & \FF           & \FF          & \FF \\
\FF & \FF & \TT & \FF           & \FF          & \FF \\
\FF & \FF & \FF & \FF           & \FF          & \FF \\
\multicolumn{3}{c}{\upbracefill} & \multicolumn{2}{c}{\upbracefill} & \multicolumn{1}{c}{\upbracefill} \\
\multicolumn{3}{c}{\begin{minipage}{40pt}\centering\scriptsize propositional\\ variables\end{minipage}} & \multicolumn{2}{c}{\begin{minipage}{40pt}\centering\scriptsize intermediate\\ subformulae\end{minipage}} & \multicolumn{1}{c}{\scriptsize main formula}
\end{tabular}
\end{center}
Alguns comentários sobre a construção desta tabela verdade são pertinentes:

\begin{itemize}
\item As variáveis ​​proposicionais aparecem primeiro. Como são três, há $2^3=8$ linhas. A coluna para $p$ contém quatro \TT{}s seguida por quatro \FF{}s; a coluna para $q$ contém dois \TT{}s, two \FF{}s, e então repete; e a coluna para $r$ contém uma \TT{}, um \FF{}, e então repete.
\item 
O próximo grupo de colunas são as próximas subfórmulas mais complicadas. Cada uma é construída observando as colunas relevantes mais à esquerda e comparando com a tabela verdade da conjunção.
\item A coluna final é a própria fórmula principal, que novamente é construída observando as colunas relevantes mais à esquerda e comparando com a tabela verdade para disjunção.
\end{itemize}
Nossas escolhas sobre onde colocar as barras verticais e em que ordem colocar as linhas não foram as únicas escolhas que poderiam ter sido feitas, mas ao construir tabelas verdade para fórmulas lógicas mais complexas, é útil desenvolver um sistema e segui-lo.
\end{example}

Returnando para \Cref{thmLogicalEquivalentIffSameTruthValues}, obtemos a seguinte estratégia para provar que duas fórmulas proposicionais são logicamente equivalentes.
\begin{strategy}[Logical equivalence using truth tables]
\label{strTruthTable}
Para provar que as fórmulas proposicionais são logicamente equivalentes, basta mostrar que elas têm colunas idênticas numa tabela verdade.
\end{strategy}

\begin{example}
Em \Cref{exConjunctionDistributesOverDisjunction} nós provamos que $p \wedge (q \vee r) \equiv (p \wedge q) \vee (p \wedge r)$. Provamos isso novamente usando tabelas verdade. Primeiro construímos a tabela verdade para $p \wedge (q \vee r)$:
\begin{center}
\begin{tabular}{ccc|c|c}
$p$ & $q$ & $r$ & $q \vee r$ & $p \wedge (q \vee r)$ \\ \hline
\TT & \TT & \TT & \TT & \TT \\
\TT & \TT & \FF & \TT & \TT \\
\TT & \FF & \TT & \TT & \TT \\
\TT & \FF & \FF & \FF & \FF \\
\FF & \TT & \TT & \TT & \FF \\
\FF & \TT & \FF & \TT & \FF \\
\FF & \FF & \TT & \TT & \FF \\
\FF & \FF & \FF & \FF & \FF
\end{tabular}
\end{center}
Observe que a coluna de $p \wedge (q \vee r)$ é idêntico ao de $(p \wedge q) \vee (p \wedge r)$ in \Cref{exFirstExampleOfTruthTable}. Portanto, as duas fórmulas são logicamente equivalentes.
\end{example}

Para evitar ter que escrever duas tabelas verdade, pode ser útil combiná-las em uma. Por exemplo, a seguinte tabela verdade mostra que $p \wedge (q \vee r)$ é logicamente equivalente a $(p \wedge q) \vee (p \wedge r)$:

\begin{center}
\begin{tabular}{ccc||c|c||cc|c}
$p$ & $q$ & $r$ & $q \vee r$ & $p \wedge (q \vee r)$ & $p \wedge q$  & $p \wedge r$ & $(p \wedge q) \vee (p \wedge r)$\\ \hline
\TT & \TT & \TT & \TT & \TT & \TT           & \TT          & \TT \\
\TT & \TT & \FF & \TT & \TT & \TT           & \FF          & \TT \\
\TT & \FF & \TT & \TT & \TT & \FF           & \TT          & \TT \\
\TT & \FF & \FF & \FF & \FF & \FF           & \FF          & \FF \\
\FF & \TT & \TT & \TT & \FF & \FF           & \FF          & \FF \\
\FF & \TT & \FF & \TT & \FF & \FF           & \FF          & \FF \\
\FF & \FF & \TT & \TT & \FF & \FF           & \FF          & \FF \\
\FF & \FF & \FF & \FF & \FF & \FF           & \FF          & \FF
\end{tabular}
\end{center}

Nos exercícios a seguir, usamos tabelas verdade para repetir as provas de equivalência lógica de \Cref{exImplicationInTermsOfDisjunction} e \Cref{exPAndQImpliesRIffPImpliesRAndQImpliesR}.

\begin{exercise}
\label{exImplicationInTermsOfDisjunctionWithTruthTables}
Use uma tabela verdade para provar que $p \Rightarrow q \equiv (\neg p) \vee q$.
\end{exercise}

\begin{exercise}
\label{exPAndQImpliesRIffPImpliesRAndQImpliesRWithTruthTables}
Let $p$, $q$ and $r$ ser variáveis ​​proposicionais. Use uma tabela verdade para provar que a fórmula proposicional $(p \vee q) \Rightarrow r$ é logicamente equivalente a $(p \Rightarrow r) \wedge (q \Rightarrow r)$.
\end{exercise}

\subsection*{Some proof strategies}

Estamos agora em boa forma para usar a equivalência lógica para derivar algumas estratégias de prova mais sofisticadas.

\begin{theorem}[Law of double negation]
\label{thmDoubleNegation}
Seja $p$ uma variável proposicional. Então $p \equiv \neg \neg p$.
\end{theorem}

\begin{cproof}
A prova é quase banalizada usando tabelas verdade. Na verdade, considere a seguinte tabela verdade.
\begin{center}
\begin{tabular}{c|c|c}
$p$ & $\neg p$ & $\neg \neg p$ \\ \hline
\TT & \FF & \TT \\
\FF & \TT & \FF
\end{tabular}
\end{center}
A coluna para $p$ and $\neg \neg p$ são idênticas, e então $p \equiv \neg \neg p$.
\end{cproof}

A lei da dupla negação é importante porque sugere uma segunda maneira de provar afirmações por contradição. Na verdade, diz que provar uma proposição $p$ é equivalente a provar $\neg \neg p$,o que equivale a assumir $\neg p$ e derivando uma contradição.

\begin{strategy}[Prova por contradição (versão indireta)]
\label{strProofByContradictionIndirect}
\index{indireta!(contradição) provado por}
\index{por contradição! prova (indireta)}
Para provar que uma proposição $p$ é verdadeira, basta assumir que $p$ é falsa e derivar uma contradição.
\end{strategy}

À primeira vista, \Cref{strProofByContradictionIndirect} parece muito parecido com \Cref{strProvingNegationsDirect}, que também denominamos \textit{prova por contradição}. Mas há uma diferença importante entre os dois:
\begin{itemize}
\item \Cref{strProvingNegationsDirect} diz que para provar que uma proposição é \textit{falso}, basta supor que é \textit{verdade} e derivar uma contradição;
\item \Cref{strProofByContradictionIndirect} diz que para provar que uma proposição é \textit{true}, basta assumir que é \textit{falso} e derivar uma contradição.
\end{itemize}

A primeira é uma técnica de prova \textit{direta}, pois surge diretamente da definição do operador de negação; a última é uma técnica de prova \textit{indireta}, uma vez que surge de uma equivalência lógica, nomeadamente a lei da dupla negação.

\begin{example}
Nós provamos que $a$, $b$ e $c$ são números reais não negativos satisfatórios $a^2+b^2=c^2$, then $a+b \ge c$.

Na verdade,se $a,b,c \in \mathbb{R}$ com $a,b,c \ge 0$, e assuma que $a^2+b^2=c^2$. Em direção a uma contradição, suponha que não seja o caso que $a+b \ge c$. Então devemos ter $a+b < c$. Mas então
\[ (a+b)^2 = (a+b)(a+b) < (a+b) c < c \cdot c = c^2\]
e então
\[ c^2 > (a+b)^2 = a^2+2ab+b^2 = c^2+2ab \ge c^2\]
Isso implica que $c^2 > c^2$, o que é uma contradição. Portanto, deve ser o caso de $a + b \ge c$, conforme necessário.
\end{example}

A próxima estratégia de prova que derivamos diz respeito à prova de implicações.

\begin{definition}
\label{defContrapositive}
\index{contrapositive}
O \textbf{contrapositivo} de uma proposição da forma $p \Rightarrow q$ é a proposta $\neg q \Rightarrow \neg p$.
\end{definition}

\begin{theorem}[Lei da contraposição]
\label{thmLawOfContraposition}
Se $p$ e $q$ são variáveis ​​proposicionais. Então $p \Rightarrow q \equiv (\neg q) \Rightarrow (\neg p)$.
\end{theorem}

\begin{cproof}
Construímos as tabelas verdade para $p \Rightarrow q$ e $(\neg q) \Rightarrow (\neg p)$.

\begin{center}
\begin{tabular}{cc||c||cc|c}
$p$ & $q$ & $p \Rightarrow q$ & $\neg q$ & $\neg p$ & $(\neg q) \Rightarrow (\neg p)$ \\ \hline
\TT & \TT & \TT & \FF & \FF & \TT \\
\TT & \FF & \FF & \TT & \FF & \FF \\
\FF & \TT & \TT & \FF & \TT & \TT \\
\FF & \FF & \TT & \TT & \TT & \TT
\end{tabular}
\end{center}

As colunas para $p \Rightarrow q$ and $(\neg q) \Rightarrow (\neg p)$ são idênticos, portanto são logicamente equivalentes.
\end{cproof}

\Cref{thmLawOfContraposition} sugere a seguinte estratégia de prova.

\begin{strategy}[Prova por contraposição]
\label{strProofByContraposition}
\index{contraposition!proof by}
\index{proof!by contraposition}
Para provar uma proposição da forma $p \Rightarrow q$, basta assumir que $q$ é falso e derivar que $p$ é falso.
\end{strategy}

\begin{example}
Fixe dois números naturais $m$ e $n$. Provaremos que se $mn > 64$, então $m>8$ ou $n>8$.

Por contraposição, basta assumir que \textit{não} é o caso de $m > 8$ ou $n > 8$, e derivar que não é o caso de $mn > 64$.

Portanto, suponha que nem $m>8$ nem $n>8$. Então $m \le 8$ e $n \le 8$, de modo que $mn \le 64$, conforme necessário.
\end{example}

\begin{exercise}
Use a lei da contraposição para provar que $p \Leftrightarrow q \equiv (p \Rightarrow q) \wedge ((\neg p) \Rightarrow (\neg q))$, e use a técnica de prova que esta equivalência sugere para provar que um número inteiro é par se e somente se seu quadrado for par.
\hintlabel{exIntegerEvenIffSquareEven}{%
Expresse esta afirmação como $\forall n \in \mathbb{Z},\, (n \text{ is even}) \Leftrightarrow (n^2 \text{ is even})$, e observe que a negação de `$x$ é par' é `$x$ é ímpar'.

}
\end{exercise}

É bom invocar resultados que parecem impressionantes como \textit{prova por contraposição}, mas na prática, a equivalência lógica entre \textit{quaisquer} duas fórmulas proposicionais diferentes sugere uma nova técnica de prova, e nem todas estas técnicas têm nomes. E, de facto, a estratégia de prova no exercício seguinte, embora útil, não tem um nome que soe engenhoso – pelo menos, não um que possa ser amplamente compreendido.

\begin{exercise}
Prove que p∨q≡(¬p)⇒qp \vee q \equiv (\neg p) \Rightarrow q. Use esta equivalência lógica para sugerir uma nova estratégia para provar proposições da forma p∨qp \vee q, e use esta estratégia para provar que se dois inteiros somam um número par, então ambos os inteiros são pares ou ambos são ímpares.
\end{exercise}

\subsection*{Negation}

Na matemática pura é comum perguntar se uma determinada propriedade é válida ou não para um objeto matemático. Por exemplo, em \Cref{secCompletenessConvergence}, veremos a convergência de sequências de números reais: para dizer que uma sequência $x_0,x_1,x_2,\dots$ de números reais \textit{converge}\label{pConvergencePreliminary} ( \Cref{defConvergenceOfSequence}) quer dizer
\[ \exists a \in \mathbb{R},\, \forall \varepsilon \in \mathbb{R},\, (\varepsilon > 0 \Rightarrow \exists N \in \mathbb{N},\, \forall n \in \mathbb{N},\, [n \ge N \Rightarrow |x_n-a| < \varepsilon])\]
Esta já é uma fórmula lógica relativamente complicada. Mas e se quiséssemos provar que uma sequência \textit{não} converge? Simplesmente assumir a fórmula lógica acima e derivar uma contradição pode funcionar às vezes, mas não é particularmente esclarecedor.

Nosso próximo objetivo é desenvolver um método sistemático para negar fórmulas lógicas complicadas. Feito isso, seremos capazes de negar a fórmula lógica que expressa “a sequência $x_0, x_1, x_2, \dots$ converges' do seguinte modo
\[ \forall a \in \mathbb{R},\, \exists \varepsilon \in \mathbb{R},\, (\varepsilon > 0 \wedge \forall N \in \mathbb{N},\, \exists n \in \mathbb{N},\, [n \ge N \wedge |x_n-a| \ge \varepsilon])\]

É verdade que esta ainda é uma expressão complicada, mas quando decomposta elemento por elemento, fornece informações úteis sobre como pode ser provada.

As regras para negar conjunções e disjunções são exemplos de \textit{Leis de Morgan}, que apresentam uma espécie de dualidade entre $\wedge$ and $\vee$.

\begin{theorem}[de Morgan's laws for logical operators]
\label{thmDeMorganLogicalOperators}
\index{de Morgan's laws!for logical operators}
Let $p$ and $q$ be logical formulae. Then:
\begin{enumerate}[(a)]
\item $\neg (p \wedge q) \equiv (\neg p) \vee (\neg q)$; and
\item $\neg (p \vee q) \equiv (\neg p) \wedge (\neg q)$.
\end{enumerate}
\end{theorem}

\begin{cproof}[of (a)]
Considere a seguinte tabela verdade.
\begin{center}
\begin{tabular}{cc||c|c||cc|c}
$p$ & $q$ & $p \wedge q$ & $\neg (p \wedge q)$ & $\neg p$ & $\neg q$ & $(\neg p) \vee (\neg q)$ \\ \hline
\TT & \TT & \TT & \FF & \FF & \FF & \FF \\
\TT & \FF & \FF & \TT & \FF & \TT & \TT \\
\FF & \TT & \FF & \TT & \TT & \FF & \TT \\
\FF & \FF & \FF & \TT & \TT & \TT & \TT \\
\end{tabular}
\end{center}
As colunas para $\neg (p \wedge q)$ and $(\neg p) \vee (\neg q)$ are identical, so they are logically equivalent.
\end{cproof}

\begin{exercise}
Prove \Cref{thmDeMorganLogicalOperators}(b) três vezes: uma vez usando a definição de equivalência lógica diretamente (como fizemos em \Cref{exConjunctionDistributesOverDisjunction,exImplicationInTermsOfDisjunction} e \Cref{exPAndQImpliesRIffPImpliesRAndQImpliesR}), uma vez usando uma tabela verdade e uma vez usando a parte (a) junto com a lei da dupla negação.
\end{exercise}

\begin{example}

Frequentemente usamos as leis de Morgan para operadores lógicos sem pensar nisso. Por exemplo, dizer que “nem $3$ nem $7$ são pares” é equivalente a dizer “$3$ é ímpar e $7$ é ímpar”. A afirmação anterior se traduz em
\[ \neg [(3 \text{ is even}) \vee (7 \text{ is even})]\]
enquanto a segunda afirmação se traduz em
\[ [\neg (3 \text{ is even})] \wedge [\neg (7 \text{ is even})]\]
\end{example}

\begin{exercise}
\label{exNegationOfImplication}
Prove que $\neg (p \Rightarrow q) \equiv p \wedge (\neg q)$ duas vezes, uma vez usando uma tabela verdade, e uma vez usando \Cref{exImplicationInTermsOfDisjunctionWithTruthTables} juntamente com as leis de de Morgan e a lei da dupla negação.
\end{exercise}

As leis de De Morgan para operadores lógicos generalizam-se para afirmações sobre quantificadores, expressando uma dualidade semelhante entre $\forall$ e $\exists$ como temos entre $\wedge$ e $\vee$.

\begin{theorem}[de Morgan's laws for quantifiers]
\label{thmDeMorganQuantifiers}
\index{de Morgan's laws!for quantifiers}
let $p(x)$ be a logical formula with free variable $x$ ranging over a set $X$. Then:
\begin{enumerate}[(a)]
\item $\neg \forall x \in X,\, p(x) \equiv \exists x \in X,\, \neg p(x)$; and
\item $\neg \exists x \in X,\, p(x) \equiv \forall x \in X,\, \neg p(x)$.
\end{enumerate}
\end{theorem}

\begin{cproof}
Infelizmente, como essas fórmulas lógicas envolvem quantificadores, não temos tabelas verdade à nossa disposição, portanto devemos assumir cada fórmula e derivar a outra.

Começamos provando a equivalência na parte (b) e então derivamos (a) como consequência.

\begin{itemize}
\item Assuma $\neg \exists x \in X,\, p(x)$. Para provar $\forall x \in X,\, \neg p(x)$, fixe um $x \in X$. If $p(x)$ fossem verdadeiras, então teríamos $\exists x \in X,\, p(x)$, o que contradiz nossa suposição principal; então temos $\neg p(x)$. Mas então $\forall x \in X,\, \neg p(x)$ é verdadeiro.

\item Assuma $\forall x \in X,\, \neg p(x)$. Por uma questão de contradição, suponha $\exists x \in X,\, p(x)$ eram verdade. Então obtemos alguns $a \in X$ para qual $p(a)$ é verdade. Mas $\neg p(a)$ é verdade pela suposição de que $\forall x \in X,\, \neg p(a)$, então obtemos uma contradição. Por isso $\neg \exists x \in X,\, p(x)$ é verdade.
\end{itemize}

Isto prova que $\neg \exists x \in X,\, p(x) \equiv \forall x \in X,\, \neg p(x)$.

Now (a) follows from (b) using the law of double negation (\Cref{thmDoubleNegation}):
\[ \exists x \in X,\, \neg p(x) \equiv \neg\neg \exists x \in X,\, \neg p(x) \overset{(b)}{\equiv} \neg \forall x \in X,\, \neg \neg p(x) \equiv \neg \forall x \in X,\, p(x)\]
as required.
\end{cproof}

A estratégia de prova sugerida pela equivalência lógica em \Cref{thmDeMorganQuantifiers}(b) é tão importante que tem nome próprio.

\begin{strategy}[Proof by counterexample]
\label{strCounterexample}
\index{counterexample}
\index{proof!by counterexample}
Para provar que uma proposição da forma $\forall x \in X,\, p(x)$ é falsa, basta encontrar um único elemento $a \in X$ de tal modo que $p(a)$ é falso. O elemento $a$ é chamado de \textbf{counterexample} para a proposta $\forall x \in X,\, p(x)$.
\end{strategy}

\begin{example}
Provamos por contra-exemplo que nem todo número inteiro é divisível por um número primo. Na verdade, seja $x = 1$. Os únicos fatores integrais de $1$ são $1$ e $-1$, nenhum dos quais é primo, de modo que $1$ não é divisível por nenhum número primo.
\end{example}

\begin{exercise}
Prove por contra-exemplo que nem todo número racional pode ser expresso como $\dfrac{a}{b}$ where $a \in \mathbb{Z}$ is even and $b \in \mathbb{Z}$ é ímpar.
\hintlabel{exNotEveryRationalIsEvenDividedByOdd}{%
Encontre um número racional cujas representações como razão de dois inteiros tenham um denominador par.
}
\end{exercise}

Vimos agora como negar os operadores lógicos $\neg$, $\wedge$, $\vee$ and $\Rightarrow$, bem como os quantificadores $\forall$ and $\exists$. 

\begin{definition}
\label{defMaximallyNegatedLogicalFormula}
\index{negation!maximal}
\index{logical formula!maximally negated}
Uma fórmula lógica é \textbf{maximally negated} se as únicas instâncias do operador de negação $\neg$ aparecem imediatamente antes de um predicado (ou outra proposição que não envolva operadores lógicos ou quantificadores).
\end{definition}

\begin{example}
A seguinte fórmula proposicional é negada ao máximo:
\[ [p \wedge (q \Rightarrow (\neg r))] \Leftrightarrow (s \wedge (\neg t))\]
Na verdade, todas as instâncias de $\neg$ aparecem imediatamente antes das variáveis ​​proposicionais.

No entanto, a seguinte fórmula proposicional \textit{não} é negada ao máximo:
\[ (\neg \neg q) \Rightarrow q\]
Aqui a subfórmula $\neg \neg q$ contém um operador de negação imediatamente antes de outro operador de negação ($\neg \neg q$). No entanto, pela lei da dupla negação, isso é equivalente a $q \Rightarrow q$, que é negado ao máximo trivialmente, uma vez que não há operadores de negação dignos de menção.
\end{example}

\begin{exercise}
Determine quais das seguintes fórmulas lógicas são negadas ao máximo.
\begin{enumerate}[(a)]
\item $\forall x \in X,\, (\neg p(x)) \Rightarrow \forall y \in X, \neg (r(x,y) \wedge s(x,y))$;
\item $\forall x \in X,\, (\neg p(x)) \Rightarrow \forall y \in X, (\neg r(x,y)) \vee (\neg s(x,y))$;
\item $\forall x \in \mathbb{R},\, [x > 1 \Rightarrow (\exists y \in \mathbb{R},\, [x < y \wedge \neg (x^2 \le y)])]$;
\item $\neg \exists x \in \mathbb{R},\, [x > 1 \wedge (\forall y \in \mathbb{R},\, [x < y \Rightarrow x^2 \le y])]$.
\end{enumerate}
\end{exercise}

O seguinte teorema permite-nos substituir fórmulas lógicas por fórmulas maximamente negadas, o que por sua vez sugere estratégias de prova que podemos usar para provar que proposições aparentemente complicadas são \textit{falsas}.

\begin{theorem}
\label{thmLogicalFormulaEquivalentToMaximallyNegated}
Cada fórmula lógica (construída usando apenas os operadores lógicos e quantificadores que vimos até agora) é logicamente equivalente a uma fórmula lógica negada ao máximo.
\end{theorem}

\begin{cidea}
Muito parecido \Cref{thmLogicalEquivalentIffSameTruthValues}, uma prova precisa de \Cref{thmLogicalFormulaEquivalentToMaximallyNegated} requer alguma forma de argumento de indução, então em vez disso daremos uma ideia da prova.

Cada fórmula lógica que vimos até agora é construída a partir de predicados usando os operadores lógicos ${\wedge}, {\vee}, {\Rightarrow}$ e $\neg$ e os quantificadores $\forall$ e $\exists$- --de fato, o operador lógico $\Leftrightarrow$ foi definido em termos de $\wedge$ e $\Rightarrow$, e o quantificador $\exists$ foi definido em termos dos quantificadores $\forall$ e $\exists$ e os operadores lógicos $\wedge$ e $\Rightarrow$.

Mas os resultados desta seção nos permitem colocar negações “dentro” de cada um desses operadores lógicos e quantificadores, conforme resumido na tabela a seguir.
\begin{center}
\begin{tabular}{rcll}
Negação fora & & Negação dentro & Prova \\ \hline
$\neg (p \wedge q)$ &$\equiv$& $(\neg p) \vee (\neg q)$ & \Cref{thmDeMorganLogicalOperators}(a) \\
$\neg (p \vee q)$ &$\equiv$& $(\neg p) \wedge (\neg q)$ &  \Cref{thmDeMorganLogicalOperators}(b) \\
$\neg (p \Rightarrow q)$ &$\equiv$& $p \wedge (\neg q)$ & \Cref{exNegationOfImplication} \\
$\neg (\neg p)$ &$\equiv$& $p$ & \Cref{thmDoubleNegation} \\
$\neg \forall x \in X,\, p(x)$ &$\equiv$& $\exists x \in X,\, \neg p(x)$ & \Cref{thmDeMorganQuantifiers}(a) \\
$\neg \exists x \in X,\, p(x)$ &$\equiv$& $\forall x \in X,\, \neg p(x)$ & \Cref{thmDeMorganQuantifiers}(b)
\end{tabular}
\end{center}

A aplicação repetida dessas regras a uma fórmula lógica eventualmente produz uma fórmula lógica logicamente equivalente e maximamente negada.
\end{cidea}

\begin{example}
Lembre-se da fórmula lógica da página \pageref{pConvergencePreliminary} expressando a afirmação de que uma sequência $x_0, x_1, x_2, \dots$  \pageref{pConvergencePreliminary} dos números reais converge:
\[ \exists a \in \mathbb{R},\, \forall \varepsilon \in \mathbb{R},\, (\varepsilon > 0 \Rightarrow \exists N \in \mathbb{N},\, \forall n \in \mathbb{N},\, [n \ge N \Rightarrow |x_n-a| < \varepsilon])\]
Negaremos isso ao máximo para obter uma fórmula lógica que expresse a afirmação de que a sequência não converge.

Vamos começar no início. A negação da fórmula com a qual começamos é:
\[ \neg \exists a \in \mathbb{R},\, \forall \varepsilon \in \mathbb{R},\, (\varepsilon > 0 \Rightarrow \exists N \in \mathbb{N},\, \forall n \in \mathbb{N},\, [n \ge N \Rightarrow |x_n-a| < \varepsilon])\]
%
A chave para negar ao máximo uma fórmula lógica é ignorar informações que não são imediatamente relevantes. Aqui, a expressão que estamos negando assume a forma $\neg \exists a \in \mathbb{R},\, (\text{stuff})$. Não importa qual seja a “coisa” ainda; tudo o que importa é que estamos negando uma afirmação existencialmente quantificada e, portanto, as leis de Morgan para quantificadores nos dizem que isso é logicamente equivalente a $\forall a \in \mathbb{R},\, \neg (\text{stuff} )$. Aplicamos esta regra e apenas reescrevemos as `coisas', para obter:
\[ \forall a \in \mathbb{R},\, \neg \forall \varepsilon \in \mathbb{R},\, (\varepsilon > 0 \Rightarrow \exists N \in \mathbb{N},\, \forall n \in \mathbb{N},\, [n \ge N \Rightarrow |x_n-a| < \varepsilon])\]
%
Agora estamos negando uma afirmação universalmente quantificada, $\neg \forall \varepsilon \in \mathbb{R},\, (\text{stuff})$ que, pelas leis de Morgan para quantificadores, é equivalente a $\exists \varepsilon \in \mathbb{R},\, \neg (\text{stuff})$:
\[ \forall a \in \mathbb{R},\, \exists \varepsilon \in \mathbb{R},\, \neg (\varepsilon > 0 \Rightarrow \exists N \in \mathbb{N},\, \forall n \in \mathbb{N},\, [n \ge N \Rightarrow |x_n-a| < \varepsilon])\]
%
Neste ponto, a afirmação que está sendo negada tem a forma $(\text{stuff}) \Rightarrow (\text{junk})$, which by \Cref{exNegationOfImplication} negates to $(\text{stuff}) \wedge \neg (\text{junk})$. Aqui, `coisas' são $\varepsilon > 0$ e `porcarias' são $\exists N \in \mathbb{N}, \forall n \in \mathbb{N},\, [n \ge N \Rightarrow |x_n - a| < \varepsilon]$. So performing this negation yields:
\[ \forall a \in \mathbb{R},\, \exists \varepsilon \in \mathbb{R},\, (\varepsilon > 0 \wedge \neg \exists N \in \mathbb{N},\, \forall n \in \mathbb{N},\, [n \ge N \Rightarrow |x_n-a| < \varepsilon])\]
%
Agora estamos negando novamente uma fórmula existencialmente quantificada, então usar as leis de Morgan para quantificadores dá:
\[ \forall a \in \mathbb{R},\, \exists \varepsilon \in \mathbb{R},\, (\varepsilon > 0 \wedge \forall N \in \mathbb{N},\, \neg \forall n \in \mathbb{N},\, [n \ge N \Rightarrow |x_n-a| < \varepsilon])\]
%
A fórmula que está sendo negada aqui é quantificada universalmente, então usar as leis de Morgan para quantificadores \textit{novamente} dá:
\[ \forall a \in \mathbb{R},\, \exists \varepsilon \in \mathbb{R},\, (\varepsilon > 0 \wedge \forall N \in \mathbb{N},\, \exists n \in \mathbb{N},\, \neg [n \ge N \Rightarrow |x_n-a| < \varepsilon])\]
%
Estamos quase lá! A afirmação negada aqui é uma implicação, portanto, aplicando a regra $\neg (p \Rightarrow q) \equiv p \wedge (\neg q)$ again yields:
\[ \forall a \in \mathbb{R},\, \exists \varepsilon \in \mathbb{R},\, (\varepsilon > 0 \wedge \forall N \in \mathbb{N},\, \exists n \in \mathbb{N},\, [n \ge N \wedge \neg (|x_n - a| < \varepsilon)])\]
%
Neste ponto, a rigor, a fórmula é negada ao máximo, uma vez que a afirmação que está sendo negada não envolve quaisquer outros operadores lógicos ou quantificadores. No entanto,desde que $\neg (|x_n-a| < \varepsilon)$ é equivalente a $|x_n - a| \ge \varepsilon$, podemos dar um passo adiante para obter:
\[ \forall a \in \mathbb{R},\, \exists \varepsilon \in \mathbb{R},\, (\varepsilon > 0 \wedge \forall N \in \mathbb{N},\, \exists n \in \mathbb{N},\, [n \ge N \wedge |x_n - a| \ge \varepsilon])\]
%
Isto é tão negado quanto poderíamos sonhar, e por isso paramos por aqui.
\end{example}

\begin{exercise}
Encontre uma fórmula proposicional maximamente negada que seja logicamente equivalente a $\neg (p \Leftrightarrow q)$.
\hintlabel{exMaximallyNegateBiconditional}{%
Comece expressando $\Leftrightarrow$ in terms of $\Rightarrow$ and $\wedge$, as in \Cref{defBiconditional}.
}
\end{exercise}

\begin{exercise}
Negue ao máximo a seguinte fórmula lógica e, em seguida, prove que é verdadeira ou prove que é falsa.
\[ \exists x \in \mathbb{R},\, [x > 1 \wedge (\forall y \in \mathbb{R},\, [x < y \Rightarrow x^2 \le y])]\]
\end{exercise}

\subsection*{Tautologies}

O conceito final que introduzimos neste capítulo é o de uma \textit{tautologia}, que pode ser pensada como o oposto de uma contradição. A palavra “tautologia” tem outras implicações quando usada coloquialmente, mas no contexto da lógica simbólica tem uma definição precisa.

\begin{definition}
\label{defTautology}
\index{tautology}
Uma \textbf{tautologia} é uma proposição ou fórmula lógica que é verdadeira, não importa como os valores de verdade são atribuídos às suas variáveis ​​​​proposicionais e predicados componentes.
\end{definition}

A razão pela qual estamos interessados ​​em tautologias é que as tautologias podem ser usadas como suposições em qualquer ponto de uma prova, por qualquer motivo.

\begin{strategy}[Assuming tautologies]
Seja $p$ uma proposição. Qualquer tautologia pode ser assumida em qualquer prova de $p$.
\end{strategy}

\begin{example}
A lei do terceiro excluído (\Cref{axLEM}) diz precisamente que $p \vee (\neg p)$ é uma tautologia. Isto significa que, ao provar qualquer resultado, podemos dividir em casos com base no facto de uma proposição ser verdadeira ou falsa, tal como fizemos em \Cref{propIfProductEvenThenSomeFactorEven}.
\end{example}

\begin{example}
S fórmula $p \Rightarrow (q \Rightarrow p)$ is a tautology.

Uma prova direta deste fato é a seguinte. Para provar que $p \Rightarrow (q \Rightarrow p)$ é verdadeiro, basta assumir $p$ e derivar $q \Rightarrow p$. Então assuma $p$. Agora, para provar $q \Rightarrow p$, basta assumir $q$ e derivar $p$. Então assuma $q$. Mas já estamos assumindo que $p$ é verdade! Portanto, $q \Rightarrow p$ é verdadeiro e, portanto, $p \Rightarrow (q \Rightarrow p)$ é verdadeiro.

Uma prova usar tabelas verdade é o seguinte:
\begin{center}
\begin{tabular}{cc|c|c}
$p$ & $q$ & $q \Rightarrow p$ & $p \Rightarrow (q \Rightarrow p)$ \\ \hline
\TT & \TT & \TT & \TT \\
\TT & \FF & \TT & \TT \\
\FF & \TT & \FF & \TT \\
\FF & \FF & \TT & \TT
\end{tabular}
\end{center}
Nós vemos que $p \Rightarrow (q \Rightarrow p)$ is true regardless of the truth values of $p$ and $q$.
\end{example}

\begin{exercise}
\label{exTautologies}
Prove que cada uma das afirmações a seguir é uma tautologia:
\begin{enumerate}[(a)]
\item $[(p \Rightarrow q) \wedge (q \Rightarrow r)] \Rightarrow (p \Rightarrow r)$;
\item $[p \Rightarrow (q \Rightarrow r)] \Rightarrow [(p \Rightarrow q) \Rightarrow (p \Rightarrow r)]$;
\item $\exists y \in Y,\, \forall x \in X,\, p(x,y) \Rightarrow \forall x \in X,\, \exists y \in Y,\, p(x,y)$;
\item $[\neg (p \wedge q)] \Leftrightarrow [(\neg p) \vee (\neg q)]$;
\item $(\neg \forall x \in X,\, p(x)) \Leftrightarrow (\exists x \in X,\, \neg p(x))$.
\end{enumerate}
Para cada um, tente interpretar o que significa e como pode ser útil em uma prova.
\end{exercise}

Você deve ter notado paralelos entre as leis de Morgan para operadores lógicos e quantificadores e as partes (d) e (e) de \Cref{exTautologias}, respectivamente. Eles quase parecem dizer a mesma coisa, exceto que em \Cref{exTautologies} usamos `$\Leftrightarrow$' e em \Cref{thmDeMorganLogicalOperators,thmDeMorganQuantifiers} usamos `$\equiv$'. Porém, há uma diferença importante: se $p$ e $q$ são fórmulas lógicas, então $p \Rightarrow q$ é em si uma fórmula lógica, que podemos estudar como um objeto matemático por si só. No entanto, $p \equiv q$ não é uma fórmula lógica: é uma afirmação \textit{sobre} fórmulas lógicas, nomeadamente que as fórmulas lógicas $p$ e $q$ são equivalentes.

Existe, no entanto, uma estreita relação entre $\Leftrightarrow$ and $\equiv$---esta relação é resumida no seguinte teorema.

\begin{theorem}
\label{thmTautologyAndDerivation}
Seja $p$ e $q$ fórmulas lógicas.
\begin{enumerate}[(a)]
\item $q$ pode ser derivado de $p$ se e somente se $p \Rightarrow q$ is a tautology;
\item $p \equiv q$ se e somente se $p \Leftrightarrow q$ for uma tautologia.
\end{enumerate}
\end{theorem}

\begin{cproof}
Para (a), observe que uma derivação de $q$ de $p$ é suficiente para estabelecer a verdade de $p \Rightarrow q$ pela regra de introdução para implicação \introrule{\Rightarrow}, e então se $q$ puder ser derivado de $p$, então $p \Rightarrow q$ é uma tautologia. Por outro lado, se $p \Rightarrow q$ é uma tautologia, então $q$ pode ser derivado de $p$ usando a regra de eliminação para implicação \elimrule{\Rightarrow} junto com a suposição (tautológica) de que $p \Rightarrow q$ é verdade.

Agora (b) segue de (a), uma vez que a equivalência lógica é definida em termos de derivação em cada direção, e $\Leftrightarrow$ é simplesmente a conjunção de duas implicações.
\end{cproof}

%\textit{Respire!} Toda essa nova notação pode ser cansativa no início, mas valerá a pena no final. Este capítulo foi sobre ensinar a você um novo idioma – novos símbolos, nova terminologia – porque sem ele, nossas atividades futuras serão impossíveis. Se você estiver travado agora, não se preocupe: em breve você pegará o jeito, especialmente quando começarmos a usar essa nova linguagem no contexto. Você pode, é claro, consultar os resultados deste capítulo para referência em qualquer momento no futuro.


\begin{tldr}{secLogicalEquivalence}
\subsubsection*{Logical equivalence}

\begin{tldrlist}
\tldritem{defLogicalEquivalence}
Duas fórmulas lógicas são \textit{logicamente equivalentes} (escritas $\equiv$) se cada uma puder ser derivada da outra.

\tldritem{defTruthTable}
A \textit{tabela verdade} de uma fórmula proposicional é uma tabulação de seus valores verdade sob todas as atribuições de valores verdade às suas variáveis ​​proposicionais constituintes. Duas fórmulas proposicionais são logicamente equivalentes se e somente se tiverem colunas idênticas em uma tabela verdade.
\end{tldrlist}

\subsubsection*{Some specific logical equivalences}

\begin{tldrlist}
\tldritem{thmDoubleNegation}
A dupla negação de uma proposição é equivalente à proposição original. Isto dá origem à prova `indireta' por contradição: para provar que uma proposição $p$ é \textit{verdadeira}, assuma que $p$ é falso e derive uma contradição.

\tldritem{defContrapositive}
A \textit{contrapositiva} de uma implicação $p \Rightarrow q$ é a implicação $(\neg q) \Rightarrow (\neg p)$. Toda implicação é equivalente à sua contrapositiva, portanto, para provar $p \Rightarrow q$, basta assumir que $q$ é falso e derivar que $p$ é falso.

\tldritem{thmDeMorganLogicalOperators}
As leis de De Morgan para operadores lógicos dizem que $\neg (p \wedge q) \equiv (\neg p) \vee (\neg q)$ and $\neg (p \vee q) \equiv (\neg p) \wedge (\neg q)$.

\tldritem{exNegationOfImplication}
A negação de $p \Rightarrow q$ is logically equivalent to $p \wedge (\neg q)$, so in order to prove that $p \Rightarrow q$ is \textit{false}, basta provar que $p$ é verdadeiro, mas $q$ é falso.

\tldritem{thmDeMorganQuantifiers}
As leis de De Morgan para quantificadores dizem que $\neg \forall x \in X,\, p(x) \equiv \exists x \in X,\, \neg p(x)$ and $\neg \exists x \in X,\, p(x) \equiv \forall x \in X,\, \neg p(x)$. A primeira delas sugere a estratégia de \textit{proof by counterexample}.
\end{tldrlist}

\subsubsection*{Maximal negation}

\begin{tldrlist}
\tldritem{defMaximallyNegatedLogicalFormula}
Uma fórmula lógica é \textit{maximamente negada} se não contém operadores de negação (exceto, talvez, imediatamente antes de uma variável ou predicado proposicional).

\tldritem{thmLogicalFormulaEquivalentToMaximallyNegated}
Cada fórmula lógica construída usando os operadores lógicos e quantificadores nesta seção é logicamente equivalente a uma fórmula maximamente negada.
\end{tldrlist}

\subsubsection*{Tautologies}

\begin{tldrlist}
\tldritem{defTautology}
Uma \textit{tautologia} é uma fórmula lógica que é verdadeira independentemente da atribuição de valores de verdade dada às suas variáveis ​​proposicionais constituintes ou de quais valores são substituídos por suas variáveis ​​livres. Tautologias podem ser invocadas como suposições em qualquer ponto de uma prova.
\tldritem{thmTautologyAndDerivation}
Uma proposição pp pode ser derivada de uma proposição qq se e somente se p⇒qp \Rightarrow q for uma tautologia; e p≡qp \equiv q se e somente se p⇔qp \Leftrightarrow q for uma tautologia.
\end{tldrlist}
\end{tldr}
